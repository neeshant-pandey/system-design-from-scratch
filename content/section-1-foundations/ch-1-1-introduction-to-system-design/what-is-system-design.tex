\section{Introduction}

System design is the process of defining the \textbf{architecture}, \textbf{components}, \textbf{modules}, \textbf{interfaces}, and \textbf{data} for a system to satisfy specified requirements. It's a critical skill for software engineers, particularly when building large-scale distributed systems.

At its core, system design is about making \textit{trade-offs}. Every design decision involves balancing competing concerns such as performance, scalability, reliability, cost, and maintainability. There is rarely a single "correct" answer—instead, we must understand the problem context and choose the most appropriate solution.

\section{Why is System Design Important?}

In today's world, systems must handle:

\begin{itemize}
\item \textbf{Massive scale}: Billions of users, petabytes of data
\item \textbf{High availability}: 99.99\% uptime requirements (less than 1 hour of downtime per year)
\item \textbf{Geographic distribution}: Users across continents with low-latency expectations
\item \textbf{Rapid evolution}: New features and requirements emerge constantly
\end{itemize}

Consider a simple example: a URL shortener like \texttt{bit.ly}. At first glance, this seems trivial—just map short URLs to long ones. But when you consider:

\begin{itemize}
\item Handling millions of requests per second
\item Ensuring shortened URLs are unique across billions of entries
\item Providing analytics on click-through rates
\item Maintaining low latency for redirects worldwide
\item Preventing abuse and malicious links
\end{itemize}

...the problem becomes significantly more complex. This is where system design thinking becomes essential.

\section{Key Characteristics of Well-Designed Systems}

\subsection{Scalability}

The ability to handle increased load by adding resources. There are two types:

\begin{enumerate}
\item \textbf{Vertical scaling (Scale-up)}: Adding more power to an existing machine (more CPU, RAM, disk)
\item \textbf{Horizontal scaling (Scale-out)}: Adding more machines to your system
\end{enumerate}

Most modern systems favor horizontal scaling because it provides better fault tolerance and is more cost-effective at large scale.

\subsection{Reliability}

The probability that a system will produce correct outputs and continue functioning over time. A reliable system:

\begin{itemize}
\item Handles faults gracefully (fault-tolerant)
\item Recovers quickly from failures
\item Maintains data consistency
\item Has no single points of failure
\end{itemize}

\subsection{Availability}

The percentage of time a system is operational and accessible. The industry uses "nines" to measure availability:

\begin{itemize}
\item 99\% availability (two nines): 3.65 days downtime/year
\item 99.9\% availability (three nines): 8.76 hours downtime/year
\item 99.99\% availability (four nines): 52.56 minutes downtime/year
\item 99.999\% availability (five nines): 5.26 minutes downtime/year
\end{itemize}

Each additional nine becomes exponentially more expensive to achieve.

\subsection{Maintainability}

The ease with which a system can be modified, updated, and operated. A maintainable system:

\begin{itemize}
\item Has clear, simple abstractions
\item Is well-documented
\item Uses monitoring and observability tools
\item Follows consistent patterns and conventions
\end{itemize}

\section{The System Design Process}

When approaching a system design problem, follow this general framework:

\begin{enumerate}
\item \textbf{Understand the requirements}
  \begin{itemize}
  \item Functional requirements (what the system should do)
  \item Non-functional requirements (how it should perform)
  \item Scale estimates (users, requests, data volume)
  \end{itemize}

\item \textbf{Define the API}
  \begin{itemize}
  \item What operations does the system expose?
  \item What are the inputs and outputs?
  \end{itemize}

\item \textbf{High-level design}
  \begin{itemize}
  \item Identify major components
  \item Define relationships between components
  \item Choose appropriate patterns (load balancer, cache, queue, etc.)
  \end{itemize}

\item \textbf{Deep dive into critical components}
  \begin{itemize}
  \item Database schema design
  \item Caching strategy
  \item Partitioning/sharding approach
  \item Replication mechanism
  \end{itemize}

\item \textbf{Identify bottlenecks and trade-offs}
  \begin{itemize}
  \item What could fail?
  \item Where are the performance bottlenecks?
  \item How can we optimize?
  \end{itemize}
\end{enumerate}

\section{Real-World Example: Designing a Rate Limiter}

Let's apply this framework to a concrete problem: designing a rate limiter that restricts users to $N$ requests per time window.

\textbf{Requirements:}
\begin{itemize}
\item Limit users to 1000 requests per hour
\item Respond with HTTP 429 (Too Many Requests) when limit exceeded
\item Support millions of users
\item Low latency (add less than 10ms overhead)
\end{itemize}

\textbf{High-level approach:}
\begin{itemize}
\item Use Redis for fast in-memory counter storage
\item Implement sliding window counter algorithm
\item Key format: \texttt{user\_id:timestamp\_bucket}
\item Set TTL on keys to auto-expire old data
\end{itemize}

\textbf{Algorithm (simplified):}

\begin{lstlisting}[language=python]
def fibonacci(n):
    """Calculate the nth Fibonacci number."""
    if n <= 1:
        return n
    return fibonacci(n-1) + fibonacci(n-2)

# Test the function
for i in range(10):
    print(f"F({i}) = {fibonacci(i)}")
\end{lstlisting}

This simple design demonstrates several key concepts:

\begin{itemize}
\item \textbf{Choosing the right data store}: Redis for low-latency atomic operations
\item \textbf{Efficient algorithm}: Sliding window with $O(1)$ time complexity
\item \textbf{Resource management}: Automatic cleanup via TTL
\item \textbf{Horizontal scalability}: Redis can be sharded across multiple nodes
\end{itemize}

\section{Common Pitfalls for Beginners}

\begin{itemize}
\item \textbf{Over-engineering}: Designing for a billion users when you have 100
\item \textbf{Ignoring trade-offs}: Every decision has costs—acknowledge them
\item \textbf{Premature optimization}: Start simple, optimize based on actual bottlenecks
\item \textbf{Single point of failure}: Always ask "what happens if this component fails?"
\item \textbf{Neglecting monitoring}: You can't fix what you can't see
\end{itemize}

\section{Conclusion}

System design is both an art and a science. It requires technical knowledge of databases, networks, caching, and distributed systems, but also the wisdom to make pragmatic trade-offs based on real-world constraints.

As you progress through this curriculum, you'll build a mental toolkit of patterns, techniques, and best practices. You'll learn when to use a SQL vs. NoSQL database, how to partition data, when to introduce caching, and how to design for failure.

Remember: \textit{The goal is not to memorize solutions, but to develop a systematic thinking process for breaking down complex problems.}

\subsection{Next Steps}

In the following topics, we'll explore:

\begin{itemize}
\item Communication patterns in distributed systems
\item Coordination and consistency models
\item Scalability techniques and architectural patterns
\item Resilience and fault tolerance strategies
\end{itemize}

Let's begin the journey into the fascinating world of system design!
