\section{What is System Design?}

System design is the process of defining the \textbf{architecture}, \textbf{components}, \textbf{modules}, \textbf{interfaces}, and \textbf{data} for a system to satisfy specified requirements. It involves making high-level decisions about how different parts of the system will interact and work together to achieve desired functionality, performance, scalability, and reliability.

\subsection{Breaking It Down}

A \textit{system} is a collection of components that work together to perform a specific function or set of functions. We have \textbf{input}, it goes through the system, and we get \textbf{output}. System design is about how we structure and organize these components to achieve our goals.

\subsubsection{Visual Representation of a System}

See the diagram below showing a simple system:

\begin{center}
\includegraphics[width=0.7\textwidth]{/assets/section-1/ch-1-1/systemdefination.png}
\end{center}

The diagram illustrates the fundamental concept: data flows in as \textit{input}, gets processed by the system's components, and produces the desired \textit{output}. This simple model applies whether you're building a calculator app or a distributed system serving millions of users.

\section{How Do We Design a System?}

Here are some key steps involved in system design:

\subsection{1. Understand the Requirements}

We start by understanding the requirements of the system. What are the \textbf{functional} and \textbf{non-functional requirements}?

\begin{itemize}
\item \textbf{Functional requirements} define what the system should do
\item \textbf{Non-functional requirements} specify how the system should perform (e.g., scalability, reliability, security)
\end{itemize}

\subsection{2. Identify Major Components}

Next, we identify the major components of the system. These could include:

\begin{itemize}
\item Databases
\item Servers
\item Client applications
\item Third-party services
\end{itemize}

We also define how these components will interact with each other through interfaces and protocols.

\subsection{3. Make Architectural Decisions}

Then, we make high-level architectural decisions. This involves choosing the overall structure of the system, such as:

\begin{itemize}
\item Whether to use a \textbf{monolithic architecture} or a \textbf{microservices architecture}
\item Deciding on key technologies and frameworks to use
\end{itemize}

\subsection{4. Focus on Specific Design Aspects}

After that, we focus on specific design aspects such as:

\begin{itemize}
\item Data storage
\item Caching strategies
\item Load balancing
\item Fault tolerance
\end{itemize}

We need to ensure that the system can handle expected loads and recover from failures gracefully.

\subsection{5. Document the Design}

Finally, we document the design decisions and create diagrams to visualize the architecture. This documentation serves as a reference for developers and stakeholders throughout the development process.

\section{Our Curriculum: A Structured Journey}

This curriculum is designed to take you from \textit{zero to expert} in system design through a carefully structured progression:

\subsection{The Learning Path}

\textbf{Foundations First (Sections I-III):} We begin with fundamentals—understanding what system design is, learning estimation techniques, and mastering database basics. You'll explore communication patterns, networking, and data management before tackling complex problems.

\textbf{Hands-On Practice (Sections IV-VI):} Once you have the foundations, you'll design real systems: URL shorteners, notification services, Netflix-like platforms, and Uber-like applications. This bridges theory and practice.

\textbf{Advanced Concepts (Sections VII-IX):} Next, we dive deep into distributed systems coordination, consistency models, advanced architectural patterns, and resiliency strategies. These are the concepts that separate senior engineers from juniors.

\textbf{Production-Ready Systems (Sections X-XI):} Finally, you'll learn about data streaming, cloud infrastructure, observability, and SRE practices—everything needed to build and operate production systems at scale.

\subsection{Why This Structure Works}

\begin{itemize}
\item \textbf{Progressive complexity}: Each section builds on previous knowledge
\item \textbf{Theory + Practice}: Every concept is reinforced with real-world system design examples
\item \textbf{Interview-ready}: Covers all major topics asked in FAANG system design interviews
\item \textbf{Production-focused}: Emphasizes patterns used in real production systems
\end{itemize}

By following this curriculum, you'll develop a \textit{systematic thinking process} for breaking down complex problems and designing scalable, reliable systems. Let's begin!
